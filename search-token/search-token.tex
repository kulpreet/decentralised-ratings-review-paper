\documentclass[a4paper]{article}

%% Language and font encodings
\usepackage[english]{babel}
\usepackage[utf8x]{inputenc}
\usepackage[T1]{fontenc}

%% Sets page size and margins
\usepackage[a4paper,top=3cm,bottom=2cm,left=3cm,right=3cm,marginparwidth=1.75cm]{geometry}

%% Useful packages
\usepackage{amsmath}
\usepackage{graphicx}
\usepackage{epstopdf}
\epstopdfsetup{update}

\usepackage[colorinlistoftodos]{todonotes}
\usepackage[colorlinks=true, allcolors=blue]{hyperref}
\usepackage{csquotes}

\usepackage{algorithm}
\usepackage{algorithmic}

\title{An incentivized search platform for searching vendors by reputation}

\author{
  Chlu
}

\begin{document}
\maketitle

\begin{abstract}

  We describe Chlu Search, a platform for searching businesses and
  service providers ranked by their reputation. The aggregation of
  ratings and reviews accumulated by vendors over their lifetime
  defines their reputation. Currently it is impossible to find vendors
  ranked by their reputation across multiple marketplaces. Fake
  reviews and walled gardens don't allow searching for vendors across
  marketplaces. Chlu Search, supported by the Chlu reputation platform
  solves this problem. The Chlu reputation platform, described in a
  separate document, captures vendor reputation so that anyone can
  validate the authenticity of reputation data without depending on a
  trusted third party and without locking the reputation data inside a
  walled garden. The Chlu Search platform, described here, builds on
  top of the Chlu reputation platform and provides incentives vendor
  and customer participation by granting Chlu Search Tokens for
  writing reviews and accepting payments. The same token is also used
  by vendors to purchase advertising space on Chlu Search. Finally, we
  specify the Chlu initial token generation event and the distribution
  of token across the founding team, investors and future generation
  of tokens that provides the incentives for vendor and customer
  participation.

\end{abstract}

\section{Introduction}

If a customer wants to search for a vendor that provides a specific
product of service, the customer has to repeat the search on multiple
marketplaces and try to correlate the various ratings and reviews on
those different marketplaces to try and select a vendor. The situation
is such because marketplaces lock down vendor reputation data into
walled gardens owned by the marketplace. For example, eBay, Amazon and
Alibaba reputation of the same vendor have no correlation. Multiple
solution have been tried on web 2.0 by Yelp, TripAdvisor and similar
platforms, to try and provide a means for customers to search for
vendors in an easy manner. However, these platforms are prone to fake
reviews as the reviews on these platforms are not backed by proof of a
purchase made by the customer. Vendors can hire fake review writers
and this results in an arms race between the review platforms and the
fake review providers.

The Chlu reputation platform\cite{chlu-reputation} describes a
solution that addresses the problem of fake reviews and allows vendors
to control their reputation data which breaks them free from the
marketplace owned walled gardens.

In this document, we describe Chlu Search, a platform that provides a
means to search the reputation data generated by the Chlu reputation
platform. The Chlu Search platform also enables incentives for vendors
to accept payments and reviews through Chlu, and or customers to write
reviews using Chlu.

Chlu Search can only enable searching for vendors who share their
reputation data with Chlu, and that is why the incentives for vendors
are important. The same is true for customers; writing reviews is
often ignored by customers, but by offering incentives Chlu Search
drives the adoption of Chlu reputation platform.

This synergy between Chlu Search and the Chlu reputation platform is
important for the ecosystem to accomplish its goals. The vendor owned
reputation data enables a future where marketplaces can't lock vendors
inside their walled gardens and the problem of fake reviews is
addressed. Meanwhile, the incentives provided by Chlu Search further
drive adoption of the Chlu reputation platform, resulting in a bigger
data set for Chlu Search to provide search for.

\section{Chlu Search}

Chlu Search enables customers to search for a potential vendor in a
domain without having to repeat the searches om multiple marketplaces
and then doing the often frustrating work of drawing parallels between
the diverse ratings systems used by the marketplaces.

Chlu Search ranks vendors by the aggregate ratings and reviews
received by vendors that are supported by verifiable Payments. The
search results therefore are based only on the ratings, 1 to 5, and
the amounts paid. There are other improvements possible, for example
using a half life on the ratings - older ratings contribute lower
scores to the vendor's rank.

The goals of Chlu Search can be defined as:

\begin{description}
\item[Search by ratings] Provide a means to search for vendors based
  on ratings received after a sale
\item[Ratings backed by payments] If there is no sale, the rating is
  considered invalid and not included in the search rank calculation
\item[Independence from marketplaces] Let vendors break free from the
  walled gardens of marketplaces, so that they can chose to start
  selling on any marketplace and even move from one to the other.
\end{description}

Chlu search is a centralised service and is not built on blockchain or
a smart contract. Search doesn't need to be decentralised. However,
the data that Chlu Search is built on is stored on a decentralised
storage network, IPFS. The reputation data provided by Chlu reputation
platform is stored on IPFS and completely under the vendor's
control. Chlu Search requires that the vendor is provided incentives
to share this data with Chlu Search. In the next section we introduce
the Chlu Search Token and show how it is used to provide vendors with
appropriate incentive.

Chlu Search also requires that the Chlu reputation platform receives
ratings data from customers. In traditional ratings systems, customers
are not given much of an incentive to create a review. Chlu Search
creates an incentive for the customers to create a review by using the
Chlu Search Token as a reward.

\section{Chlu Search Token}

As described above the Chlu Search Token serves the important function
of providing incentives to the vendor and the customer so that there
is data generated for Chlu Search to be relevant and there are enough
vendors listed on Chlu Search. In this section, we describe the
various uses of the Chlu Search Token and provide the initial details
of the Chlu Search Token economy.

\subsection{Providing Incentives for creating reviews with Chlu}

Chlu Search rewards customers who create reviews on the Chlu
reputation platform. The reward is in the form of Chlu Search Tokens
that can be converted to other cryptocurrencies.

The only requirements to earn such rewards are

\begin{enumerate}
\item The customer is registered to receive the reward and
\item The vendor is sharing the ratings data with Chlu Search
\end{enumerate}

The first requirement is not required a priori. Instead, the customer
can make payments and leave ratings for a vendor and come back much
later to collect the reward for the work. As soon as a customer
registers with Chlu and proves they are the authors of their ratings
and reviews, the customer is rewarded with Chlu Search Tokens as per
the specification described later.

To prove that a customer is the author of reviews, the customer wallet
includes a signature in the review record and the public key to
validate this signature is shared with Chlu Search at the time of user
registration. See \cite{chlu-repuation} for details of the review
record. The signature defined here is not required by the Chlu
reputation record and therefore is not yet specified there.

The second requirement listed above states that if the vendor is not
sharing the ratings data with Chlu Search, the customer will not be
rewarded for leaving the review. Chlu wallets can explicitly
communicate this information to the customer, so that the customer is
aware if they will receive a reward for leaving a review. Customers
might even chose to not do business with a vendor who is not sharing
their reputation data with Chlu simply because the customer wants to
receive the reward of Chlu Search Tokens.

\subsection{Providing Incentives for Increased Adoption}

Chlu Search rewards both the customer and the vendor to participate in
the Chlu economy.

\begin{enumerate}
\item The customer is rewarded for writing reviews
\item The vendor is rewarded for sharing their ratings data with Chlu
  Search
\end{enumerate}

With the right schedule of rewards we want to drive adoption of both
the Chlu reputation platform and Chlu Search, while at the same time,
allowing vendors to control their own reputation data and not being
confined inside a walled garden run by a marketplace or a ratings
platform.

Up till now we have described how the Chlu Search Token is generated
as a reward to be given to vendors and customers. In the next section
we describe how the Chlu Search Token is consumed in the Chlu economy.

\subsection{Advertising on Chlu Search}

Vendors are allowed to advertise within the Chlu Search platform. So a
plumber in New York can purchase ad space, essentially boosting their
profile in the search results by paying for ad space in Chlu Search
Tokens. These spent Chlu Search Tokens are recycled and included in
the rewards for vendors and customers.

With the generation of tokens for incentives and consumption of tokens
for Chlu Search, we model a token economy so that the two are balanced
and the token economy remains balanced.

\subsection{Chlu Token Economy}

Before we describe how the Chlu Token economy is maintained, we need
to specify how many tokens will exist, i.e. the cap on the number of
Chlu Search tokens. We have decided this cap to be 2.1 billion
tokens. At the time of token generation event, 50\% of the tokens will
be generated and sold to initial token buyers. The remaining 50\% of
the tokens will be generated or ``mined'' as rewards for customers and
vendors when they write reviews or share reviews with Chlu Search,
respectively.

With the cap on Chlu Search Tokens defined we can now describe how the
Chlu Token economy will work. While balancing the token economy we
have the following variables available to us:

\begin{enumerate}
\item How many reviews were created in a given time period
\item How many Chlu Search Tokens were spent in purchasing ad space on
  Chlu Search
\item If the Chlu Token cap has not been reached, then how many tokens
  should be generated in the given time period.
\end{enumerate}

We mention a ``time period'' above, and for the moment we consider
this to be an hour. So that every hour new tokens are generated and
rewarded to vendors and customers. If there are no new reviews in a
given hour, then no rewards are to be given out, so no new Chlu Tokens
are generated. If instead there are reviews and there have been ad
purchases in a given hour, then the rewards for that hour will include
both the tokens.

\begin{algorithm}
  \caption{Chlu Search Token Reward Distribution}
  \label{reward-algo}
  \begin{algorithmic}
    \STATE $cap\_reached \leftarrow$ Have all the tokens have already been generated?
    \STATE $num\_reviews \leftarrow$ Number of reviews created in the given period
    \STATE $add\_spend \leftarrow$ Number of tokens spent purchasing ad space in the given period
    \IF{$cap\_reached$ is false and $num\_reviews >= 1$}
    \STATE generate $6000$ Chlu Search Tokens
    \STATE distribute amongst new review authors
    \ELSIF{$num\_reviews >= 1$ and $add\_spend >= 1$}
    \STATE distribute $add\_spend$ tokens as reward amongst new review authors
    \STATE Chlu keeps a 5\% of $add\_spend$ as  commission for providing the Search platform
    \ENDIF
  \end{algorithmic}    
\end{algorithm}

Pseudo code in Algorithm \ref{reward-algo} shows how Chlu Search Tokens
are generated and once the maximum number of tokens have been
generated, the rewards are derived from the amount of tokens spent
purchasing ad space on Chlu Search.

\subsection{Token Generation Event}

The maximum number of Chlu Search Tokens ever generated will not
exceed 2.1 billion, out of which 50\% are for sale during the token
generation event to fund the development of the Chlu reputation and
search platforms.

5\% of the tokens are set aside for providing incentives to
marketplaces for integratin Chlu reputation and payments
platform. 30\% of the tokens are not generated at the time of the the
token sale and instead are generated as reviews are created and
customers and vendors have to be rewarded. Finally, 15\% of the tokens
are reserved for the founders.

\begin{center}
  \begin{table}
    \begin{tabular}{l r r}
      \textbf{Purpose} & \textbf{When generated} & \textbf{Allocation (\%)} \\
      \hline \\
      ICO Sale & Genesis & 50 \\
      Marketplaces, partners, giveaway  & Genesis & 5 \\
      Chlu Founders retain & Genesis & 15 \\
      Customer and vendor incentives & Reward & 30
    \end{tabular}
    \caption{Token distribution}
  \end{table}
\end{center}

\subsection{Token reward rate}

Chlu Search Tokens will be generated at a maximum rate of 6000 tokens
per hour. If there are no reviews created in a given hour, then no
tokens are generated. If there are one or more reviews then the 6000
tokens are distributed between the customers who created the reviews
and the vendors who shared their review with Chlu Search.

\subsubsection{Sybil attacks}

There is a potential attack where users could create a marketplace to
generate a proof of payment request and then pay themselves using a
sock puppet account. These attacks are discussed in
\cite{chlu-reputation} where we also show how these attacks can be
turned useless by marketplaces refusing to consider reviews and
ratings that appear to be from non-reputable marketplaces.

To thwart such attacks Chlu token rewards are given to customers and
vendors only if they register with Chlu Search to receive the rewards
and have validated their identity using any on of the decentralised
identity platforms like Civic\cite{civic} or
Blockstack\cite{blockstack}.

There is no minimum payment required to receive Chlu Search Token as
reward for writing or receiving reviews. Further, there is no
correlation between the amounts paid and the reward received. To drive
adoption all payments are considered equal on purpose, we want the
market to decide where platforms like the Chlu reputation platform and
Chlu Search become popular.

\section{Conclusion}

We described how Chlu Search builds on top of the Chlu reputation
platform and enables searches for vendors across marketplaces. We
showed how Chlu Search provides incentives for customers to create
ratings and reviews for vendors by rewarding them with Chlu Search
Tokens. We also showed how vendors are provided incentives with Chlu
Search Tokens for sharing their reputation data with Chlu Search.

Finally we described the distribution of Chlu Search Tokens during the
initial token generation event and how more tokens are generated as
adoption increases. We also showed how the token economy works once
all tokens have been generated

\medskip
 
\begin{thebibliography}{9}

\bibitem{chlu-reputation} Chlu, \textit{Reviews and Ratings Verified
  by Payments on a Blockchain}. 2017.
  \url{https://chlu.io/papers/position-paper.pdf}

\bibitem{civic} Chlu, \textit{Civic}. 2017.
  \url{https://www.civic.com/}

\bibitem{blockstack} Chlu, \textit{Blockstack}. 2017.
  \url{https://blockstack.org/}
  
\end{thebibliography}

\end{document}
