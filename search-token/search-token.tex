\documentclass[a4paper]{article}

%% Language and font encodings
\usepackage[english]{babel}
\usepackage[utf8x]{inputenc}
\usepackage[T1]{fontenc}

%% Sets page size and margins
\usepackage[a4paper,top=3cm,bottom=2cm,left=3cm,right=3cm,marginparwidth=1.75cm]{geometry}

%% Useful packages
\usepackage{amsmath}
\usepackage{graphicx}
\usepackage{epstopdf}
\epstopdfsetup{update}

\usepackage[colorinlistoftodos]{todonotes}
\usepackage[colorlinks=true, allcolors=blue]{hyperref}
\usepackage{csquotes}

\title{An incentivized search platform for searching vendors by reputation}

\author{
  Chlu
}

\begin{document}
\maketitle

\begin{abstract}

  We describe Chlu Search, a platform for searching businesses and
  service providers ranked by their reputation. The aggregation of
  ratings and reviews accumulated by vendors over their lifetime
  defines their reputation. Currently it is impossible to find vendors
  ranked by their reputation across multiple marketplaces. Fake
  reviews and walled gardens don't allow searching for vendors across
  marketplaces. Chlu Search, supported by the Chlu reputation platform
  solves this problem. The Chlu reputation platform, described in a
  separate document, captures vendor reputation so that anyone can
  validate the authenticity of reputation data without depending on a
  trusted third party and without locking the reputation data inside a
  walled garden. The Chlu Search platform, described here, builds on
  top of the Chlu reputation platform and incentivizes vendor and
  customer participation by granting Chlu Search Tokens for writing
  reviews and accepting payments. The same token is also used by
  vendors to purhase advertising space on Chlu Search. Finally, we
  specify the Chlu initial token generation event and the distribution
  of token across the founding team, investors and future generation
  of tokens that provides the incentives for vendor and customer
  participation.

\end{abstract}

\section{Introduction}

If a customer wants to search for a vendor that provides a specific
product of service, the customer has no choice but to repeat the
search on multiple marketplaces and try to correlate the various
ratings and reviews on those different marketplaces to try and select
a vendor. The situation is such because marketplaces lock down vendor
reputation data into walled gardens owned by the marketpalce. For
example, eBay, Amazon and Alibaba reputation of the same vendor have
no correlation. Multiple solution have been tried on web 2.0 by Yelp,
TripAdvisor and similar platforms, to try and provide a means for
customers to search for vendors in an easy manner. However, these
platforms are prone to fake reviews as the reviews on these platforms
are not backed by proof of a purchase made by the customer. Vendors
can hire fake review writers and this results in an arms race between
the review platforms and the fake review providers.

The Chlu reputation platform\cite{chlu-reputation} describes a
solution that addresses the problem of fake reviews and allows vendors
to control their reputation data which breaks them free from the
marketplace owned walled gardens.

In this document, we describe Chlu Search, a platform that provides a
means to search the reputation data generated by the Chlu reputation
platform. The Chlu Search platform also enables incentives for vendors
to accept payments and reviews through Chlu, and or customers to write
reviews using Chlu.

Chlu Search can only enable searching for vendors who share their
reputation data with Chlu, and that is why the incentives for vendors
are important. The same is true for customers, writing reviews is
often ignored by customers, but by offering incentives Chlu Search
drives the adoption of Chlu reputation platform.

This synergy between Chlu Search and Chlu reputation platform is
important for the ecosystem to accomplish its goals. The vendor owned
reputation data enables a future where marketplaces can't lock in
vendors inside their walled gardens and the problem of fake reviews is
addressed. Meanwhile, the incentives provided by Chlu Search further
drive adoption of Chlu reputation platform resulting in bigger data
set for Chlu Search to provide search for.

\section{Chlu Search}

Chlu Search enables customers to search for a potential vendor in a
domain without having to repeat the searches om multiple marketplaces
and then doing the often frustrating work of drawing parallels between
the diverse ratings systems used by the marketplaces.

Chlu Search ranks vendors by the aggregate ratings and reviews
received by vendors that are supported by validatable payments. The
search results therefore are based only on the ratings, 1 to 5, and
the amounts paid. There are other improvements possible, for example
using a half life on the ratings - older ratings contribute lower
scores to the vendor's rank.

The goals of Chlu Search can defined as:

\begin{enumerate}
\item[Search by ratings] Provide a means to search for vendors based
  on ratings received after a sale
\item[Ratings backed by payments] If there is no sale, the rating is
  considered invalid and not included in the search rank calculation
\item[Independency from marketplaces] Let vendors break free from the
  walled gardens of marketplaces, so that they can chose to start
  selling on any marketplace and even move from one to the other.
\end{enumerate}

Chlu search is a centralised service and is not built on blockchain or
a smart contract. Search doesn't need to be decentralised. However,
the data that Chlu Search is built on is stored on a decentralised
storage network, IPFS. The reputation data provided by Chlu reputation
platform is stored on IPFS and completely under the vendor's
control. Chlu Search requires that the vendor is incentivized to share
this data with. In the next section we introduce the Chlu Search Token
and show how it is used to provide vendors with appropriate incentive.

Chlu Search also needs that the Chlu reputation platform receives
ratings data from customers. In traditional ratings systems, customers
are not given much of an incentive to create a review. Chlu Search
creates an incentive for the customers to create a review by using the
Chlu Search Token as a reward.

\section{Chlu Search Token}

As described above the Chlu Search Token serves the important function
of incentivizing the vendor and the customer so that there is data
generated for Chlu Search to be relevant and there are enough vendors
listed on Chlu Search. In this section, we describe the various uses
of the Chlu Search Token and provide the initial details of the Chlu
Search Token economy.

\subsection{Incentivize creating reviews with Chlu}

Chlu Search rewards customers who create reviews on the Chlu
reputation platform. The reward is in the form of Chlu Search Tokens
that can be converted to other cryptocurrencies.

The only requirements to earn such rewards is that

\begin{enumerate}
\item The customer is registered to receieve the reward and
\item The vendor is sharing the ratings data with Chlu Search
\end{enumerate}

The first requirement is not required apriori. In stead, the customer
can make payments and leave ratings for a vendor and come back much
later to collect the reward for the work. As soon as a customer
registers with Chlu and proves they are the authors of their ratings
and reviews, the customer is rewarded with Chlu Search Tokens as per
the specification described later.

[TODO - change Chlu reputation protocol so that customer reviews are
  created under their IPNS directory]

To prove that a customer is the author of reviews, the customer wallet
creates a validation signature in the IPNS directory where the reviews
are saved. By registering with Chlu and proving their authorship of
the reviews, customers can receive reward Chlu Search Tokens in an
asynchronous manner.

The second requirement listed above states that if the vendor is not
sharing the ratings data with Chlu Search, the customer will not be
rewarded for leaving the review. Chlu wallets can explicitly
communicate this information to the customer, so that the customer is
aware if they will receive a reward for leaving a review. Customers
might even chose to not do business with a vendor who is not sharing
their reputation data with Chlu simply because the customer wants to
recieve the reward of Chlu Search Tokens.

\subsection{Incentivizing Adoption}

Chlu Search rewards both the customer and the vendor to participate in
the Chlu economy.

\begin{enumerate}
\item The customer is rewarded for writing reviews
\item The vendor is rewarded for sharing their ratings data with Chlu
  Search
\end{enumerate}

With the right schedule of rewards we want to drive adoption of both
the Chlu reputation platform and Chlu Search. At the same time,
allowing vendors to control their own reputation data and not being
confined inside a walled garden run by a marketplace or a ratings
platform.

Up till now we have described how the Chlu Search Token is generated
as a reward to be given to vendors and customers. In the next section
we describe how the Chlu Search Token is consumed in the Chlu economy.

\subsection{Pay for advertising on Chlu Search}

Vendors are allowed to advertise within the Chlu Search platform. So a
plumber in New York can purchase ad space, essentially bumping his
profile to the top of the search results by spending Chlu Search
Tokens. These spent Chlu Search Tokens are transferred to Chlu, the
company running the Chlu Search Token.

[TODO - Describe precisely when generation stops, and what are the
  generation schedules and later the rewards schedule based on ad spends]

Once all the Chlu Search Tokens have been generated by the reward generation

\section{Chlu Token Economy}

- Token cap
- Rate of token generation
- Token distribution


\section{Conclusion}


\medskip
 
\begin{thebibliography}{9}

\bibitem{ipfs} 
  Benet, Juan
  \textit{IPFS - Content Addressed, Versioned, P2P File System}. 2014.
  \url{https://github.com/ipfs/papers/raw/master/ipfs-cap2pfs/ipfs-p2p-file-system.pdf}

\bibitem{openbazaar} 
  \textit{Decentralized Reputation in OpenBazaar}
  \newline
  \url{https://blog.openbazaar.org/decentralized-reputation-in-openbazaar}

\bibitem{ethlance}
  \textit{Ethlance - hire or work for Ether cryptocurrency}
  \url{https://ethlance.com}

\bibitem{district0x}
  \textit{A collective of decentralized marketplaces and communities}
  \url{https://district0x.io/docs/district0x-whitepaper.pdf}

\bibitem{colony}
  \textit{Colony}
  \url{https://colony.io}

\bibitem{monetha}
  \textit{Monetha}
  \url{https://www.monetha.io/static/media/Monetha_White_Paper.72d6c2dc.pdf}
  
\end{thebibliography}

\end{document}
